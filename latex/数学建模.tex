\documentclass{article}
\usepackage{forest,ctex,amsmath,color,xcolor,colortbl,graphicx,geometry,comment,listings,hyperref,booktabs,tabularx,wrapfig,caption,calc,mhchem,fix-cm,tocloft,titlesec,setspace,fancyhdr,booktabs,float,subcaption}
\usepackage{lastpage}
\usepackage[absolute,overlay]{textpos}  % 绝对定位
\setlength{\parindent}{2em} 
% ========== 标题格式定制(titlesec宏包) ========== %
\renewcommand{\thesection}{\chinese{section}}  % 一级标题编号:第X章
\renewcommand{\thesubsection}{\arabic{section}.\arabic{subsection}} % 二级:1.1
\renewcommand{\thesubsubsection}{\arabic{section}.\arabic{subsection}.\arabic{subsubsection}} % 三级:1.1.1
% 一级标题:三号黑体,居中,上下各空12pt(≈1行)
\titleformat{\section}
  [block]                  % 标题格式:display(带上下间距)
  {\centering \zihao{3} \heiti}  % 居中,三号黑体
  {\Large \thesection}       % 编号格式(Large字号,如“第一章”)
  {12pt}                     % 编号与标题的间距
  {}                         % 标题内容(无额外格式)
\titlespacing{\section}{0pt}{30pt}{30pt}  % 上间距12pt,下间距12pt

% 二级标题:四号黑体,缩进,编号显示(如“1.1”)
\titleformat{\subsection}
  {\zihao{4} \heiti}         % 四号黑体
  {\thesubsection}           % 编号(如“1.1”)
  {1em}                      % 编号与标题的间距
  {}
\titlespacing{\subsection}{0em}{1.5ex plus .5ex}{1ex plus .2ex}  % 左缩进2em

% 三级标题:小四号黑体,缩进,编号显示(如“1.1.1”)
\titleformat{\subsubsection}
  {\zihao{-4}}        % 小四号黑体(\zihao{-4}对应小四号)
  {\thesubsubsection}        % 编号(如“1.1.1”)
  {1em}                      % 编号与标题的间距
  {}
\titlespacing{\subsubsection}{0em}{1.5ex plus .5ex}{1ex plus .2ex}  % 左缩进2em


\geometry{left=2.5cm,right=2.5cm,top=2.5cm,bottom=2.5cm}
%\setlength{\parindent}{2pt} 
\date{}
\begin{document}
\vspace{-30pt}
\begin{center}
    \Large \textbf{数学建模比赛}  % 原字体可能是\Large,改为\large缩小
\end{center}
% 标题与“摘要”之间减少空白
\vspace{-13pt}
\begin{center}
    \large \textbf{摘要}  % 原字体可能是\large,改为\normalsize缩小
\end{center}
\qquad 本文针对NIPT(无创产前检测)技术中最佳检测时点选择与胎儿异常判定的问题,通过建立数学模型,对
孕妇进行合理分组并确定最佳检测策略,以最小化潜在风险并提高检测准确性。我们综合利用了数理统计等方法,
系统地解决了题目提出的四个问题。

针对问题一,我们对于NIPT检测中胎儿Y染色体浓度与孕妇孕周数和BMI等指标的相关特性问题,建立了\textbf{非
    线性混合效应模型}进行分析。通过处理附件提供的孕妇NIPT数据,考虑了个体差异(如BMI、孕周)对Y染
色体浓度的影响。模型采用\textbf{指数形式}描述Y染色体浓度随孕周的变化,并引入BMI作为线性项。使用\textbf{最大似然
    估计拟合模型},参数显著性检验表明所有固定效应均高度显著(p值<0.001)。模型结果显示,Y染色体浓度
随孕周增加呈指数增长,而与BMI呈负相关。该模型为孕妇分组和最佳NIPT时点选择提供了科学依据,有助
于提高检测准确性并降低治疗窗口期风险。

针对问题二,本研究基于临床已知的男胎孕妇BMI是影响胎儿Y染色体浓度达标时间的主要因素,
通过对孕妇BMI进行科学分组并确定最佳NIPT检测时点,以最小化孕妇潜在风险。首先,我们依据Y染色体浓度
达标情况将孕妇分为三组:始终达标组、中间达标组和从不达标组;接着采用\textbf{K-means聚类算法}对孕妇平均BMI
进行聚类分析,通过\textbf{肘部法则}确定最佳聚类数为5类;然后计算每个BMI区间的Y染色体浓度达标时间,以中位数
确定该组最佳检测时点;最后通过\textbf{误差传播模型}分析检测误差对结果的影响。结果表明,基于BMI分组的个性化
检测时点推荐能够有效降低临床风险,为NIPT临床实践提供了科学依据。

针对问题三

针对问题四

\textbf{关键词:非线性混合模型,最大似然估计混合模型,K-means聚类算法,肘部法则,误差传播模型}
\newpage
\section{\textbf{问题重述}}
\subsection{\textbf{问题背景}}
NIPT(Non-invasive Prenatal Testing,无创产前检测)作为一项革命性的产前筛查技术,通过采集孕
妇外周血中的胎儿游离DNA进行测序分析,能够有效评估胎儿常见染色体非整倍体异常的风险[1]。该技术具
有无创、安全、准确性高等特点,已成为临床产前筛查的重要手段。

在实际临床应用中,NIPT检测的准确性受到多种因素影响,其中胎儿游离DNA浓度(特别是Y染色体浓度对
于男胎)是关键因素之一。临床实践表明,胎儿Y染色体浓度与孕妇孕周数及身体质量指数(BMI)存在显著
相关性。同时,检测时机的选择对于尽早发现胎儿异常、降低临床风险至关重要:早期发现(12周以内
)风险较低,中期发现(13-27周)风险较高,而晚期发现(28周以后)风险极高。

目前临床通常根据孕妇BMI值进行简单分组并确定统一的检测时点,但这种方法未能充分考虑孕妇年龄、体
重等个体差异,可能导致部分孕妇错过最佳检测时机,增加临床风险。因此,需要建立更加科学、个性化的
NIPT时点选择模型,为不同特征的孕妇群体制定最优检测策略。

\subsection{\textbf{问题要求}}
附件提供了某地区(大多为高BMI)孕妇的NIPT检测数据,包括孕妇年龄、BMI、孕周数、胎儿染色体浓度、
Z值、GC含量、读段数等相关指标。现需要根据这些数据建立数学模型,解决以下问题:

\textbf{问题一:}基于附件中的数据,分析胎儿Y染色体浓度与孕妇孕周数、BMI等指标的相关特性,建
立合适的数学模型描述它们之间的关系,并对模型的显著性进行统计检验。

\textbf{问题二:}临床证明男胎孕妇的BMI是影响胎儿Y染色体浓度达标时间(浓度≥4\%的最早时间)的
主要因素。请对男胎孕妇的BMI进行合理分组,确定每组的BMI区间和最佳NIPT检测时点,使得孕妇的潜在
风险最小,并分析检测误差对结果的影响。

\textbf{问题三:}综合考虑体重、年龄等多种因素对男胎Y染色体浓度达标时间的影响,同时考虑检测误
差和胎儿Y染色体浓度达标比例,根据男胎孕妇的BMI进行合理分组,确定每组的最佳NIPT检测时点,使孕
妇潜在风险最小,并分析检测误差对结果的影响。

\textbf{问题四:}针对女胎异常的判定问题,以女胎孕妇的21号、18号和13号染色体非整倍体为判定结
果,综合考虑X染色体及上述染色体的Z值、GC含量、读段数及相关比例、BMI等因素,建立女胎异常的判定
模型和方法。

\section{\textbf{问题分析}}
\subsection{\textbf{问题一的分析}}
针对问题一,其要求分析胎儿Y染色体浓度与孕妇孕周数和BMI等指标的相关特性,并建立关系模型。NIPT
检测的准确性依赖于Y染色体浓度(对于男胎),而浓度受孕周和BMI等因素影响。数据来自多个孕妇,存在
重复测量和个体差异,因此需要采用混合效应模型处理随机效应。挑战在于选择合适的模型形式(线性或非
线性)并检验显著性。本文首先进行探索性数据分析,观察Y染色体浓度与孕周和BMI的关系,发现非线性趋
势。然后尝试多种模型形式,最终采用非线性混合效应模型,包含随机截距和随机斜率,以捕捉个体变异。
模型显著性通过固定效应的t检验和p值评估。
\subsection{\textbf{问题二的分析}}
针对问题二,其要求根据男胎孕妇的BMI进行合理分组并确定最佳检测时点。我们根据Y染色体浓度是否达
标(≥4\%)将孕妇分为三类,反映不同孕妇的Y染色体浓度变化规律。然后采用K-means聚类算法对孕妇平均BMI
进行客观分组,避免主观划分的随意性。通过肘部法则确定最佳聚类数为5,确保了分组的科学性和合理性。
接着对于每个BMI分组,计算组内孕妇的Y染色体浓度达标时间(对于中间达标组采用插值法预测),以中位数
作为该组的最佳检测时点,减少异常值影响。最后考虑检测过程中可能存在的测量误差和模型误差,通过误差
传播模型评估这些误差对达标时间预测的影响程度,确保推荐时点的可靠性。
\subsection{\textbf{问题三的分析}}

\subsection{\textbf{问题四的分析}}

\section{\textbf{模型假设}}
1.假设附件所提供的孕妇NIPT检测数据真实、准确,且数据样本足以反映胎儿染色体浓度与孕妇孕周、BMI等指标间的统计规律。

2.假设题目中给出的Y染色体浓度4\% 的临界值是可靠且普适的。

3.假设具有相似特征(如处于同一BMI区间)的孕妇群体,其胎儿Y染色体浓度的增长规律和达标时间具有相似的统计特征。

4.假设题目所提供的特征(包括但不限于X、21、18、13号染色体的Z值、GC含量、读段数比例及孕妇BMI等)包含了足以有效判别女胎染色体是否异常的信息。
\section{\textbf{符号说明}}
\begin{table}[htbp]
    \centering
    % 列格式:第一列居中+固定宽3cm,第二列居中
    \begin{tabular*}{\linewidth}{@{\extracolsep{\fill}}>{\centering\arraybackslash}p{3cm} c}
        \toprule  % 顶部粗线
        符号 & 说明 \\
        \midrule  % 表头与内容间的细线
        $k_i^{(c)}$ & 孕妇代码为$i$的个体Y染色体浓度与孕周线性回归方程斜率 \\
        $k_i^{(b)}$ & 孕妇代码为$i$的个体BMI指数与孕周线性回归方程斜率  \\
        $Y_{ij}$ & 孕妇代码为$i$的个体第$j$次检测得到的Y染色体浓度\\
        $G_{ij}$ & 孕妇代码为$i$的个体第$j$次检测的检测孕周(天为单位)\\
        $BMI_i$ & 表示第$i$个孕妇的身体质量指数(常数对于每个孕妇)\\
        $\overline{b}$ & 所有孕妇个体BMI增长率的平均数\\
        \bottomrule  % 底部粗线
    \end{tabular*}
    \label{tab:symbols}
\end{table}
%%%%%%%%%%%%%%%%%%%%%%%%%%%%%%%%%%%%%%%%%%%%%%%%%%%%%%%%%%%%%%%%%%%%%%%%%%%%%%%%%%%%
%%%%%%%%%%%%%%%%%%%%%%%%%%%%%%%%%%%%%%%%%%%%%%%%%%%%%%%%%%%%%%%%%%%%%%%%%%%%%%%%%%%%
%                                   QUESTION ONE                                   %
%%%%%%%%%%%%%%%%%%%%%%%%%%%%%%%%%%%%%%%%%%%%%%%%%%%%%%%%%%%%%%%%%%%%%%%%%%%%%%%%%%%%
%%%%%%%%%%%%%%%%%%%%%%%%%%%%%%%%%%%%%%%%%%%%%%%%%%%%%%%%%%%%%%%%%%%%%%%%%%%%%%%%%%%%
\section{\textbf{问题一模型建立求解}}
\subsection{\textbf{模型建立思路}}
问题一要求分析胎儿Y染色体浓度与孕妇孕周数和BMI等指标的相关特性。我们先观察通过分析数据,对每个
人线性回归,发现了绝大多数人y和时间关系与y和增长率和BMI增长率关系,以此我们判定,所有曲线都遵
循一个共同的、内在的生物学规律,但由于个体差异,曲线的高度和形状会各不相同。为精确刻画个体增
长规律并避免群体平均带来的巨大偏差,我们决定采用基于个体时间序列的混合效应方法。核心思路是:首
先对数据进行筛选,保证每个个体的数据点数量足以支持回归分析,;然后通过使用三种线性与非线性混合效
应模型,比较得出最优是GAMM,并且根据这个得出得出胎儿Y染色体浓度与孕妇孕周数和BMI的相关特性。
\subsection{\textbf{数据预处理}}
为探究Y染色体浓度与孕妇孕周数和BMI的关系,我们分别以检测孕周(天数)和BMI为x轴,Y染色体浓度为y轴,
绘制了散点图(其中相同颜色的点代表同一个人),如下:
% 单个图片
\begin{figure}[H]  % [H]表示强制当前位置(可选参数:h=此处,t=顶部,b=底部,p=单独页)
    \centering  % 图片居中
    % 插入图片:width=0.8\textwidth 表示占页面宽度的80%(可调整)
    \includegraphics[width=0.6\textwidth]{graph/sss.png}  % 替换为实际图片路径
    \caption{y染色体浓度散点图}  % 图片标题
    \label{fig:single}  % 标签(用于交叉引用:\ref{fig:single})
\end{figure}
分析左图(Y染色体浓度随孕周变化)可见,随着孕周从80天增加到200天,Y染色体浓度整体呈上升趋势,
散点分布呈现明显的"向右上方扩散"特征。特别是在孕周大于160天后,高浓度点(Y$\geq$0.15)的出现频率
显著增加,这表明Y染色体浓度与孕周之间存在正相关关系

分析右图(Y染色体浓度与BMI关系)显示,随着BMI值在20-40范围内变化,Y染色体浓度的散点分布未呈现
明显的单向变化趋势。然而,结合后续模型得到的负系数估计值(c = -0.0042),提示Y染色体浓度可能
随BMI增加而轻微下降,但这种趋势在散点图中不够明显,可能需要通过模型来进一步验证。

附件中的数据存在个别孕妇检测次数过少的情况,这会导致回归结果不可靠。为确保模型稳定性,我们
设定了数据筛选条件:仅保留检测次数大于3次的孕妇数据。
\subsection{\textbf{线性回归模型}}
分析胎儿 Y 染色体浓度与孕妇的孕周数和 BMI 等指标的相关特性,本研究为每位孕妇建立了双变量时间
序列回归模型。从预处理后的有效数据中,筛选出每位孕妇多次检测的记录,分别建立以下两个一元线性
回归模型:

\textbf{1. Y染色体浓度-时间关系模型:}\\
以检测时间($t_{ij}$)为自变量,对应的Y染色体浓度($c_{ij}$)为因变量,建立回归模型:
\begin{gather}
    c_{ij}=k_i^{(c)}*t_{ij}+b_i^{(c)}+\epsilon_{ij}^{(c)} \tag{1}
\end{gather}

其中:$k_i^{(c)}$ 表示第 $i$ 位孕妇的胎儿Y染色体浓度的日增长率,这是我们关注的核心参数;
$b_i^{(c)}$ 为截距项;$\epsilon_{ij}^{(c)}$ 为随机误差项。

\textbf{2. BMI-时间关系模型:}\\
以检测时间($t_{ij}$)为自变量,对应的Y染色体浓度($c_{ij}$)为因变量,建立回归模型:
\begin{gather}
    c_{ij}=k_i^{(b)}*t_{ij}+b_i^{(b)}+\epsilon_{ij}^{(b)} \tag{2}
\end{gather}

我们采用最小二乘法进行参数估计,并计算确定系数 $R^2_i$ 以评估拟合优度
\subsection{\textbf{Y染色体浓度与时间关系的个体分析}}
基于建立的所有个体回归模型,我们首先对模型的关键参数进行了统计分析,结果如下表所示:
\begin{table}[htbp]
    \centering
    \begin{tabular*}{\linewidth}{@{\extracolsep{\fill}}c c c c}
        \toprule  % 顶部粗线
        统计量 & 斜率 $k_i$(\%/天) & 截距        & 确定系数$R_i^2$ \\
        \midrule  % 表头与内容间的细线
        平均值 & 0.000800       & -0.005467 & 0.861443    \\
        标准差 & 0.00316        & 0.038959  & 0.123487    \\
        最小值 & 0.001752       & 0.088208  & 0.999747    \\
        最大值 & 0.000205       & -0.151320 & 0.505933    \\
        \bottomrule  % 底部粗线
    \end{tabular*}
    \label{tab:crops_booktabs}
\end{table}

从表中可知,个体的Y染色体浓度均随孕天增加呈现明显的上升趋势,绝大部分模型的拟合优度较高(平均
$R^2=0.824$表明线性模型能够很好地描述浓度随时间变化的规律。
\subsection{\textbf{BMI增长率与Y染色体浓度增长率的全局相关性分析}}
为探究BMI是否导致了上述增长速率的差异,我们计算了每位孕妇的BMI增长率个体Y染色体浓度增长率 $k_i$ 的相关性。

我们绘制了标准化BMI增长速率与斜率 $k_i$ 在相应编号下的散点图
与两者在不同维度下绘制的图像,如下
%两个图并排
\begin{figure}[H]
    \centering
    % 子图1:宽度占页面的45%(左右留空)
    \begin{subfigure}[b]{0.45\textwidth}  % [b]表示底部对齐
        \centering
        \includegraphics[width=\textwidth]{graph/points.png}  % 宽度=子图宽度
        \label{fig:sub1}  % 子图标签
    \end{subfigure}
    \hspace{0.05\textwidth}  % 两图间距(5%页面宽度)
    % 子图2
    \begin{subfigure}[b]{0.45\textwidth}
        \centering
        \includegraphics[width=\textwidth]{graph/linear_regression.png}
        \label{fig:sub2}
    \end{subfigure}
    \label{fig:two}  % 整体标签
\end{figure}

经过线性拟合,发现$R^2$教下,拟合效果较差,因此寻找别的模型拟合。

\subsection{\textbf{模型选择思路}}
\textbf{从生物学机理出发}:

题目背景与临床知识表明,胎儿游离DNA浓度(此处体现为Y染色体浓度)随孕周增加而增长。这是因为随着
胎儿发育,其细胞凋亡释放到母体血液中的DNA总量增加。这种增长通常不是线性的,在早期增长缓慢,中后
期可能加速,符合指数增长或逻辑增长的生物学规律。因此,我们优先考虑非线性模型
(如指数模型 $a\cdot e^{b\cdot t}$)来捕捉孕周与浓度的关系。

而对于BMI与Y染色体的浓度关系,较高的BMI可能意味着更多的血液量和脂肪组织,理论上会对胎儿游离DNA浓
度产生一定的“稀释”效应。因此,我们假设BMI对Y染色体浓度存在一个线性的负向影响(即 $c\cdot BMI$ 项,
且预期$c$为负值)。将这一项加入模型,符合我们的生理学直觉。

\textbf{从数据结构出发}:

我们的数据集包含对同一孕妇的多次观测。这意味着数据点之间并非独立,存在组内相关性。忽略这种相关
性而使用普通最小二乘法回归,会严重违反模型独立性假设,导致参数估计偏差和标准误低估。

以上两点共同决定了\textbf{混合效应模型}(Mixed-Effects Model) 是处理该数据最合适的框架。它
通过引入随机效应(Random Effects)来刻画每个孕妇独有的截距和/或斜率,从而准确分离组内和组间变
异,得到更稳健、更可靠的固定效应(Fixed Effects)估计值
\subsection{\textbf{非线性混合模型建立}}
由以上分析,我们建立如下非线性混合效应模型:
\begin{gather}
    Y_{ij}=a_i\cdot e^{b_i\cdot G_i}+c\cdot BMI_i \tag{1}
\end{gather}
\begin{itemize}
    \item $a_i$和$b_i$是随机效应,随个体变化,假设服从多元正态分布:$[a_i,b_i]^T~N([a,b]^T,\sum)$,
          其中$\sum$为方差-协方矩阵。
    \item $c$ 是固定效应参数,表示BMI对Y染色体浓度的线性影响。
    \item $\epsilon_{ij}$是残差项,假设独立同分布,服从正态分布:$\epsilon_{ij}~N(0,\aleph^2)$
\end{itemize}
固定效应部分包括$a$,$b$和$c$,随机效应部分包括$a_i$和$b_i$的随机截距和斜率。
\subsection{\textbf{模型求解}}
使用R语言中的nlme包进行模型拟合。由于非线性混合效应模型对初始值敏感,我们尝试了多组初始值,最终选择初始值
$a=0.1, b=0.001, c=0.0001$。模型拟合采用最大似然估计(ML)。
拟合结果如下:
\begin{itemize}
    \item 固定效应估计:
          \begin{table}[htbp]
              \centering
              \begin{tabular*}{\linewidth}{@{\extracolsep{\fill}}c c c c c}
                  \toprule  % 顶部粗线
                  系数    &   Value  &  Std.Error & t-value & p-value  \\
                  \midrule  % 表头与内容间的细线
                  $a$ & 0.15411242 & 0.008214096  & 18.76195 &  0.000..    \\
                  $b$ & 0.00268888 & 0.000172746  & 15.56555  &  0.000..   \\
                  $c$ & -0.00422844 & 0.000273517  & -15.45948 & 0.000..    \\
                  \bottomrule  % 底部粗线
              \end{tabular*}
              \caption{系数计算}
              \label{tab:crops_booktabs}
          \end{table}
    \item 随机效应标准差和相关性:\\
          $a_i$的标准差:0.032293091  \\
          $b_i$的标准差:0.001832759  \\
          $a_i$和$b_i$的相关系数:-0.81  \\
    \item 残差标准差:0.013848063
    \item AIC = -5213.352, BIC = -5178.446, Log-Likelihood = 2613.676
\end{itemize}
所有固定效应参数的p值均小于0.001,表明参数高度显著。随机效应显示个体间存在变异,且a和b呈负相关。
\subsection{\textbf{模型检验与解释}}
\textbf{显著性检验:}

我们采用t检验对模型各参数进行显著性检验,其统计量为:
\begin{gather}
    t=\frac{\beta}{SE(\beta)} \tag{2}
\end{gather}
从我们的模型结果可得:
\begin{itemize}
    \item 参数a的t值为18.76 (df=813, p<0.001)
    \item 参数b的t值为15.57 (df=813, p<0.001)
    \item 参数c的t值为-15.46 (df=813, p<0.001)
\end{itemize}

所有参数的p值均小于0.001,表明孕周和BMI对Y染色体浓度的影响在统计上高度显著。

\textbf{参数置信区间:}

基于t分布,我们计算各参数的95\%置信区间:
$\beta \pm t_{0.975,813}\cdot SE(\beta)$\\
其中$t_{0.975, 813} \approx 1.963$,计算得到:
\begin{itemize}
    \item 参数a的95\%置信区间:$0.1541\pm1.963\times0.00821 = [0.138, 0.170]$
    \item 参数b的95\%置信区间:$0.002689\pm1.963\times0.000173 = [0.00235, 0.00303]$
    \item 参数c的95\%置信区间:$-0.004228\pm1.963\times0.000274 = [-0.00476, -0.00369]$
\end{itemize}

所有置信区间均不包含0,进一步证实了参数的统计显著性。

\textbf{残差分析:}

我们基于已有的标准化残差结果进行分析:
\begin{itemize}
    \item 残差标准差:0.01385
    \item 标准化残差分布:Q1 = -0.176, 中位数 = 0.370, Q3 = 0.824
    \item 范围:[-4.59, 4.86]
\end{itemize}

虽然大部分残差分布在合理范围内,但存在个别异常值(绝对值$\geq$4),这可能源于测量误差或个体特异因素。
\subsection{\textbf{最终方程式}}
模型方程可写为:
\begin{gather}
    Y=0.1541\cdot e^{0.002689\cdot G}-0.004228\cdot BMI \tag{3}
\end{gather}
其中
\begin{itemize}
    \item 指数项系数b = 0.002689 $\geq$ 0,证实Y染色体浓度随孕周增加呈指数增长,这与胎儿发
          育过程中游离DNA释放量增加的生物学机制一致
    \item BMI系数c = -0.004228 $\leq$ 0,表明Y染色体浓度随BMI增加而线性下降,符合BMI较高
          孕妇血液量增加产生的"稀释效应"
    \item 随机效应分析显示,个体间存在显著变异($a_i$方差=$0.001043$,$b_i$方差=$3.36×10^{-6}$),
          且a与b呈负相关(r=-0.81),提示基础浓度较高的孕妇其浓度增长速率相对较慢
\end{itemize}

注:上述方程为总体平均模型,针对个体预测时需加入随机效应项:$a_i = 0.1541 + u_i$, $b_i = 0.002689 + v_i$,其中$u_i$和$v_i$为个体随机效应。
%%%%%%%%%%%%%%%%%%%%%%%%%%%%%%%%%%%%%%%%%%%%%%%%%%%%%%%%%%%%%%%%%%%%%%%%%%%%%%%%%%%%
%%%%%%%%%%%%%%%%%%%%%%%%%%%%%%%%%%%%%%%%%%%%%%%%%%%%%%%%%%%%%%%%%%%%%%%%%%%%%%%%%%%%
%                                   QUESTION TWO                                   %
%%%%%%%%%%%%%%%%%%%%%%%%%%%%%%%%%%%%%%%%%%%%%%%%%%%%%%%%%%%%%%%%%%%%%%%%%%%%%%%%%%%%
%%%%%%%%%%%%%%%%%%%%%%%%%%%%%%%%%%%%%%%%%%%%%%%%%%%%%%%%%%%%%%%%%%%%%%%%%%%%%%%%%%%%
\section{\textbf{问题二模型建立求解}}
\subsection{\textbf{数据预处理}}
本问题旨在探究孕妇BMI对胎儿Y染色体浓度最早达标时间的影响。考虑到同一孕妇的BMI随时间变化幅度较小
(个体内变异),且个体间差异远大于时间变化量,为提高分析的准确性和可靠性,我们采用每位孕妇多次BMI
测量的平均值作为其代表值进行后续统计分析。

为合理区分样本并便于计算NIPT最佳时点,我们依据Y染色体浓度达标情况对孕妇进行了科学分组:
\begin{itemize}
    \item 第一组\textbf{(始终达标组)}:从首次检测开始Y染色体浓度即达到或超过4\%的样本。该组数据保存在 `bmi\_Y\_always\_can\_test\_result.xlsx`中,包含孕妇代码、平均BMI和最早达标天数(即首次检测时间)。
    \item 第二组\textbf{(中间达标组)}:初始检测未达标但后续检测达标的样本。该组数据保存在 \\
          `bmi\_Y\_middle\_result.xlsx`中,包含孕妇代码、平均BMI和预测达标天数(通过\textbf{插值法}计算获得,详见附录代码Line XX)。
    \item 第三组\textbf{(从不达标组)}:所有检测中Y染色体浓度均未达标的样本。该组数据保存在 \\
          `bmi\_Y\_cannot\_test\_result.xlsx`中,包含孕妇代码、平均BMI和最晚达标天数(即末次检测时间)。
\end{itemize}

为确保数据质量,若某样本曾达标但后续检测又低于4\%,则视为无效样本并予以剔除。经上述处理,最终获得
有效样本总量为230例。其中,第一组186例(占比80.87\%),第二组37例(占比16.09\%),第三组7例(占
比3.04\%)。这一分组策略确保了后续分析的可靠性和代表性。

\subsection{\textbf{BMI划分与聚类分析}}
为探究BMI对Y染色体浓度达标时间的影响模式,我们采用聚类算法对孕妇平均BMI进行科学细分。首先,通过肘
部法则确定最佳聚类数量(图1)。结果显示,当聚类数超过5时,距离平方和下降速率显著降低,故选择聚类
数k=5作为最优解。
% 单个图片
\begin{figure}[H]  % [H]表示强制当前位置(可选参数:h=此处,t=顶部,b=底部,p=单独页)
    \centering  % 图片居中
    % 插入图片:width=0.8\textwidth 表示占页面宽度的80%(可调整)
    \includegraphics[width=0.8\textwidth]{graph/zhoubu.png}  % 替换为实际图片路径
    \caption{肘部法则图}  % 图片标题
    \label{fig:single}  % 标签(用于交叉引用:\ref{fig:single})
\end{figure}

基于此,我们对孕妇平均BMI与Y染色体浓度进行了K-means聚类分析,结果如图2所示。聚类结果清晰地展示
了不同BMI区间与Y染色体浓度的关联模式。
% 单个图片
\begin{figure}[H]  % [H]表示强制当前位置(可选参数:h=此处,t=顶部,b=底部,p=单独页)
    \centering  % 图片居中
    % 插入图片:width=0.8\textwidth 表示占页面宽度的80%(可调整)
    \includegraphics[width=0.8\textwidth]{graph/julei1.png}  % 替换为实际图片路径
    \caption{聚类结果散点图}  % 图片标题
    \label{fig:single}  % 标签(用于交叉引用:\ref{fig:single})
\end{figure}

\subsection{\textbf{聚类分类评价}}
为进一步评估聚类质量,我们计算了三个内部评价指标:
\begin{itemize}
    \item \textbf{轮廓系数}:0.551(轮廓系数取值范围[-1,1],值越大表示同类样本越相近,不同类样本越远离,一般不可能大于0.7 )
          \begin{gather}
              s(i)=\frac{b(i)-a(i)}{\max{a(i),b(i)}} \tag{5}
          \end{gather}
          \( a(i) \): 样本 \( i \) 到\textbf{同一簇内}所有其他样本的平均距离。\textbf{(凝聚度)}
          \textbf{\( b(i) \)}: 样本 \( i \) 到\textbf{其他某个簇}的所有样本的平均距离。计算样本 \( i \)
          与所有\textbf{非本身所在簇}的的平均距离,然后取其中的\textbf{最小值}。\textbf{(分离度)}

    \item \textbf{DBI指数}:0.517(值越小表示类内紧密度越高,类间分离度越好,<0.7可接受 )
          \begin{gather}
              DBI = \frac{1}{K} \sum_{i=1}^{K} \max_{j \neq i} R_{ij} \tag{6}
          \end{gather}
          K: 簇的个数。$R_{ij}$: 簇 $i$ 和簇 $j$ 之间的\textbf{相似度}$s_i$: 簇 $i$ 的簇内散度,即簇内所有样本到其质心$c_i$的平均距离。
          $d_{ij}$: 簇 $i$ 的质心$ c_i$ 与簇$j$ 的质心 $c_j$ 之间的距离(通常使用欧氏距离)。

    \item \textbf{CH指数}:701.814(值越大表示聚类效果越优 )
          \begin{gather}
              CH = \frac{\text{SS}_B / (K - 1)}{\text{SS}_W / (N - K)} \tag{7}
          \end{gather}
          N: 数据集中的总样本数。K: 簇的个数。$\text{SS}_B$: \textbf{簇间离散度}(Between-Cluster Sum
          of Squares)。所有簇的质心 $c_k$ 与全局质心 $c$ 的距离平方和,并按簇大小加权。
          $\text{SS}_W$: \textbf{簇内离散度}(Within-Cluster Sum of Squares)。所有样本点到其所属簇的质心的距离平方和。

          \subsection{\textbf{NIPT时点计算}}

\end{itemize}
上述指标综合表明,本次聚类效果良好,各类别内部样本相似度高,类别之间差异明显,分类结果可靠有效,为后续分析奠定了坚实基础。
\subsection{\textbf{NIPT时点计算}}
\subsubsection{\textbf{方法原理与模型构建}}
为确定各BMI分组的最佳NIPT检测时点,本研究构建了一个基于风险-效用权衡的优化模型。该模型综合考虑了早期检测与晚期检测的双重风险,通过数学优化方法寻找使总体风险最小化的检测时间点。

\textbf{风险函数定义:}
\begin{itemize}
    \item 早期检测风险$(R_{early})$:检测时间过早可能导致Y染色体浓度尚未达标,造成假阴性结果。我们采用指数衰减函数模拟这一风险:
          \[
              R_{\text{early}}(t) = 2.0 \cdot e^{-t/50} \tag{1}
          \]
          该函数从初始值2.0开始,随孕期增加呈指数衰减,反映早期预测的高风险性。
    \item 晚期检测风险$(R_{late})$:检测时间过晚则可能错过最佳干预时机。我们采用分段函数描述这一风险:
          \[
              R_{\text{late}}(t) =
              \begin{cases}
                  0.1                                                  & t < 84             \\
                  0.1 + 0.9 \cdot \left( \frac{t-84}{189-84} \right)^2 & 84 \leq t \leq 189 \\
                  1.0                                                  & t > 189
              \end{cases} \tag{2}
          \]
          该函数在84天前保持低风险(0.1),84-189天间以二次函数形式增长,189天后达到最高风险(1.0)。
\end{itemize}
\textbf{效用函数构建:}

基于累积达标概率F(t)(通过Kaplan-Meier估计器计算)和上述风险函数,我们构建综合效用函数:
\[
    U(t) = (1 - F(t)) \cdot R_{\text{early}}(t) + F(t) \cdot R_{\text{late}}(t)\\
\]
其中F(t)表示在时间t之前Y染色体浓度达到4\%的累积概率。

\textbf{优化目标:}

通过最小化效用函数U(t)确定最佳检测时点:
\[
    t^* = \arg\min_{t \in [0, T_{\text{max}}]} U(t) \\
\]
其中$T_{max}$为观察期上限。

\subsubsection{\textbf{计算流程与实施}}
本研究采用以下步骤计算各BMI分组的最佳NIPT时点:

1. \textbf{数据准备与分组}:基于聚类分析得到的5个BMI区间,将样本分为相应组别。

2. \textbf{生存分析}:对每个BMI分组,应用Kaplan-Meier方法估计累积达标函数$F(t)$,处理右删失数据(始终达标组和从不达标组)。

3. \textbf{效用函数计算}:在时间域$[0, T_{max}]$上离散采样,计算各时间点的效用值$U(t)$。

4. \textbf{优化求解}:采用有界优化算法(Bounded方法)寻找使$U(t)$最小化的时间点t*。

5. \textbf{误差分析}:通过蒙特卡洛模拟(100次重复)评估检测误差对最优时点稳定性的影响。
\subsubsection{结果与分析}
表2展示了各BMI分组的最佳NIPT时点及相应指标:
\begin{table}[htbp]
    \centering
    \caption{各BMI分组最佳NIPT时点计算结果}
    \begin{tabular*}{\linewidth}{@{\extracolsep{\fill}}c c c c c}
        \toprule  % 顶部粗线
        BMI分组 & 最优时点(天) & 最小效用值 & 风险水平 & 样本量 \\
        \midrule  % 表头与内容间的细线
        $\leq$30.26 & 89.5 & 0.1299 & 7.696559&59 \\
        30.26-32.30&90.8&0.1360&7.3511&68 \\
        32.30-34.92&90.8&0.1359&7.3591&69\\
        34.92-39.49&99.4&0.1569&6.3753&33\\
        $\geq$39.49&110.7&0.1884&5.3090&4\\
        \bottomrule  % 底部粗线
    \end{tabular*}
    \label{tab:crops_booktabs}
\end{table}
结果表明,随着BMI增加,最佳检测时点相应延后,而效用值逐渐增加(风险水平降低)。这一趋势与临床观察
一致,高BMI孕妇需要更长时间等待Y染色体浓度达标,但一旦达标,其检测风险相对较低。

图3展示了BMI分组"34.92-39.49"的典型分析结果,包括累积达标曲线、效用函数曲线及最优时点标注。
\begin{figure}[H]
    \centering
    % 子图1:宽度占页面的45%(左右留空)
    \begin{subfigure}[b]{0.35\textwidth}  % [b]表示底部对齐
        \centering
        \includegraphics[width=\textwidth]{graph/BMI_lt30.26_utility_analysis.png}  % 宽度=子图宽度
        \caption{小于30.26组}  % 子标题
        \label{fig:sub1}  % 子图标签
    \end{subfigure}
    \hspace{0.05\textwidth}  % 两图间距(5%页面宽度)
    % 子图2
    \begin{subfigure}[b]{0.35\textwidth}
        \centering
        \includegraphics[width=\textwidth]{graph/BMI_30.26_32.30_utility_analysis.png}
        \caption{30.26-32.30组}
        \label{fig:sub2}
    \end{subfigure}
    \label{fig:two}  % 整体标签
\end{figure}
\begin{figure}[H]
    \centering
    % 子图1:宽度占页面的45%(左右留空)
    \begin{subfigure}[b]{0.35\textwidth}  % [b]表示底部对齐
        \centering
        \includegraphics[width=\textwidth]{graph/BMI_32.30_34.92_utility_analysis.png}  % 宽度=子图宽度
        \caption{32.30-34.92组}  % 子标题
        \label{fig:sub1}  % 子图标签
    \end{subfigure}
    \hspace{0.05\textwidth}  % 两图间距(5%页面宽度)
    % 子图2
    \begin{subfigure}[b]{0.35\textwidth}
        \centering
        \includegraphics[width=\textwidth]{graph/BMI_34.92_39.49_utility_analysis.png}
        \caption{34.92-39.49组}
        \label{fig:sub2}
    \end{subfigure}
    \label{fig:two}  % 整体标签
\end{figure}
% 单个图片
\begin{figure}[H]  % [H]表示强制当前位置(可选参数:h=此处,t=顶部,b=底部,p=单独页)
    \centering  % 图片居中
    % 插入图片:width=0.8\textwidth 表示占页面宽度的80%(可调整)
    \includegraphics[width=0.35\textwidth]{graph/BMI_gt39.49_utility_analysis.png}  % 替换为实际图片路径
    \caption{大于39.49组}  % 图片标题
    \label{fig:single}  % 标签(用于交叉引用:\ref{fig:single})
\end{figure}
\subsubsection{\textbf{误差敏感性分析}}
为评估检测误差对结果稳定性的影响,我们进行了蒙特卡洛模拟分析。假设检测误差服从正态分布(标准差=
5\%),对每个BMI分组进行100次重复模拟。

表3a和表3b展示了误差分析结果:
\begin{table}[htbp]
    \centering
    \caption{\textbf{表3a:检测误差对最优时点的影响分析(时点相关指标)}}
    \begin{tabular*}{\linewidth}{@{\extracolsep{\fill}}c c c c c}
        \toprule  % 顶部粗线
        BMI分组 & 原始最优时点(天) & 模拟最优时点均值(天)&时点标准差(天)& 95\%置信区间(天)\\
        \midrule  % 表头与内容间的细线
        <30.26 & 89.51&90.05 & 1.56 & (89.74, 90.35)\\
        30.26-32.30 & 90.79 & 92.64 & 2.52 & (92.14, 93.13)\\
        32.30-34.92 & 90.76 & 91.19 & 1.39 & (90.92, 91.46)\\
        34.92-39.49 & 99.44 & 100.04 & 2.37 & (99.58, 100.51)\\
        >39.49 & 110.75 & 110.11 & 3.49 & (109.43, 110.79)\\
        \bottomrule  % 底部粗线
    \end{tabular*}
    \label{tab:crops_booktabs}
\end{table}
\begin{table}[htbp]
    \centering
    \setlength{\tabcolsep}{2pt}  % 列间距从6pt缩到2pt(可调整)
    \small  % 字体缩小一级(可选:\footnotesize 更小)
    \caption{\textbf{表3b:检测误差对最优时点的影响分析(效用值与风险水平相关指标)}}
    \begin{tabular*}{\linewidth}{@{\extracolsep{\fill}}c c c c c c c}
        \toprule  % 顶部粗线
        BMI分组&原始最小效用值&模拟最小效用值均值&效用值标准差&原始风险水平&模拟风险水平均值&风险水平标准差\\
        \midrule  % 表头与内容间的细线
        $\leq$30.26&0.1299&0.1274&0.0023&7.6965&7.8546&0.1448\\
        30.26-32.30&0.1360&0.1322&0.0032&7.3511&7.5677&0.1808\\
        32.30-34.92&0.1359&0.1336&0.0021&7.3591&7.4851&0.1199\\
        34.92-39.49&0.1569&0.1594&0.0027&6.3753&6.2773&0.1072\\
        $\geq$39.49&0.1884&0.1829&0.0124&5.3090&5.4973&0.4362\\
        \bottomrule  % 底部粗线
    \end{tabular*}
    \label{tab:crops_booktabs}
\end{table}
结果表明:

1. 检测误差对最优时点的影响较小(时点变异系数<3.5\%),模型具有较好的稳健性。

2. 高BMI分组对检测误差更为敏感,时点波动范围更大。

3. 所有分组的95\%置信区间范围合理,验证了推荐时点的可靠性。

4. 效用值和风险水平的变异程度相对较小,表明模型对这些指标的预测较为稳定。

图4展示了误差模拟的详细结果,包括时点分布、效用值分布及风险分布。
\begin{figure}[H]
    \centering
    % 子图1:宽度占页面的45%(左右留空)
    \begin{subfigure}[b]{0.35\textwidth}  % [b]表示底部对齐
        \centering
        \includegraphics[width=\textwidth]{graph/error_analysis_BMI__30.26.png}  % 宽度=子图宽度
        \caption{小于30.26组}  % 子标题
        \label{fig:sub1}  % 子图标签
    \end{subfigure}
    \hspace{0.05\textwidth}  % 两图间距(5%页面宽度)
    % 子图2
    \begin{subfigure}[b]{0.35\textwidth}
        \centering
        \includegraphics[width=\textwidth]{graph/error_analysis_BMI_30.26-32.30.png}
        \caption{30.26-32.30组}
        \label{fig:sub2}
    \end{subfigure}
    \label{fig:two}  % 整体标签
\end{figure}
\begin{figure}[H]
    \centering
    % 子图1:宽度占页面的45%(左右留空)
    \begin{subfigure}[b]{0.35\textwidth}  % [b]表示底部对齐
        \centering
        \includegraphics[width=\textwidth]{graph/error_analysis_BMI_32.30-34.92.png}  % 宽度=子图宽度
        \caption{32.30-34.92组}  % 子标题
        \label{fig:sub1}  % 子图标签
    \end{subfigure}
    \hspace{0.05\textwidth}  % 两图间距(5%页面宽度)
    % 子图2
    \begin{subfigure}[b]{0.35\textwidth}
        \centering
        \includegraphics[width=\textwidth]{graph/error_analysis_BMI_34.92-39.49.png}
        \caption{34.92-39.49组}
        \label{fig:sub2}
    \end{subfigure}
    \label{fig:two}  % 整体标签
\end{figure}
% 单个图片
\begin{figure}[H]  % [H]表示强制当前位置(可选参数:h=此处,t=顶部,b=底部,p=单独页)
    \centering  % 图片居中
    % 插入图片:width=0.8\textwidth 表示占页面宽度的80%(可调整)
    \includegraphics[width=0.35\textwidth]{graph/error_analysis_BMI__39.49.png}  % 替换为实际图片路径
    \caption{大于39.49组}  % 图片标题
    \label{fig:single}  % 标签(用于交叉引用:\ref{fig:single})
\end{figure}

\subsubsection{\textbf{创新与亮点}}
本研究在NIPT时点计算方面具有以下创新点:

1. \textbf{综合风险建模}:首次将早期检测风险与晚期检测风险同时纳入考量,构建了更符合临床实际的效用函数。

2. \textbf{生存分析方法应用}:采用Kaplan-Meier方法处理右删失数据,提高了累积达标概率估计的准确性。

3. \textbf{蒙特卡洛误差分析}:通过模拟检测误差的影响,量化了模型的稳健性,为临床决策提供了可靠性评估。

4. \textbf{BMI分层优化}:针对不同BMI群体分别计算最优时点,实现了个性化检测推荐。

本研究提供的NIPT时点推荐方案,不仅考虑了检测准确性,还综合评估了时间相关的风险因素,为临床实践提
供了科学依据。结果表明,基于BMI分层的个性化检测时点选择,能够有效降低孕妇的潜在风险,提高检测效率。
%%%%%%%%%%%%%%%%%%%%%%%%%%%%%%%%%%%%%%%%%%%%%%%%%%%%%%%%%%%%%%%%%%%%%%%%%%%%%%%%%%%%
%%%%%%%%%%%%%%%%%%%%%%%%%%%%%%%%%%%%%%%%%%%%%%%%%%%%%%%%%%%%%%%%%%%%%%%%%%%%%%%%%%%%
%                                 QUESTION THREE                                   %
%%%%%%%%%%%%%%%%%%%%%%%%%%%%%%%%%%%%%%%%%%%%%%%%%%%%%%%%%%%%%%%%%%%%%%%%%%%%%%%%%%%%
%%%%%%%%%%%%%%%%%%%%%%%%%%%%%%%%%%%%%%%%%%%%%%%%%%%%%%%%%%%%%%%%%%%%%%%%%%%%%%%%%%%%
\section{\textbf{问题三模型建立求解}}
%%%%%%%%%%%%%%%%%%%%%%%%%%%%%%%%%%%%%%%%%%%%%%%%%%%%%%%%%%%%%%%%%%%%%%%%%%%%%%%%%%%%
%%%%%%%%%%%%%%%%%%%%%%%%%%%%%%%%%%%%%%%%%%%%%%%%%%%%%%%%%%%%%%%%%%%%%%%%%%%%%%%%%%%%
%                                 QUESTION FOUR                                    %
%%%%%%%%%%%%%%%%%%%%%%%%%%%%%%%%%%%%%%%%%%%%%%%%%%%%%%%%%%%%%%%%%%%%%%%%%%%%%%%%%%%%
%%%%%%%%%%%%%%%%%%%%%%%%%%%%%%%%%%%%%%%%%%%%%%%%%%%%%%%%%%%%%%%%%%%%%%%%%%%%%%%%%%%%
\section{\textbf{问题四模型建立求解}}
\section{\textbf{参考文献}}
 [1] Scientific Platform Serving for Statistics Professional 2021. SPSSPRO.
(Version 1.0.11)[Online Application Software]. Retrieved from https://www.spsspro.com.

[2] \text{Saroj,Kavita.}Review:study on simple k mean and modified K mean clustering
technique[J].International Journal of Computer Science Engineering and Technology,2016,6(7):279-281.

[3] Kaplan, E. L.; Meier, P. (1958). "Nonparametric estimation from incomplete
observations". Journal of the American Statistical Association. 53 (282): 457–481.

[4] Virtanen, P. et al. (2020). "SciPy 1.0: Fundamental Algorithms for Scientific
Computing in Python". Nature Methods. 17: 261–272.

[5] Davidson-Pilon, C. (2019). "Lifelines: Survival Analysis in Python". Journal
of Open Source Software. 4(40): 1317.
\end{document}
\begin{comment}

% 分段函数 Y_{tijk}
\[
    Y_{tijk} =
    \begin{cases}
        0, & \text{第 } i \text{ 季度未在第 } j \text{ 块地上种植物 } k \\
        1, & \text{第 } i \text{ 季度在第 } j \text{ 块地上种植物 } k
    \end{cases}
    \tag{4}
\]
\end{comment}


\begin{comment}
%公式
\begin{gather}
    X_{tijk} \leq M \cdot Y_{tijk} \quad \forall t,i,j,k \tag{5} \\
    X_{tijk} \geq 0.01 \cdot Y_{tijk} \quad \forall t,i,j,k \tag{6}
\end{gather}
\end{comment}


\begin{comment}

% 单个图片
\begin{figure}[H]  % [H]表示强制当前位置(可选参数:h=此处,t=顶部,b=底部,p=单独页)
    \centering  % 图片居中
    % 插入图片:width=0.8\textwidth 表示占页面宽度的80%(可调整)
    \includegraphics[width=0.8\textwidth]{数学建模image/屏幕截图2025-08-04185706.png}  % 替换为实际图片路径
    \caption{单张示例图片(如实验装置图)}  % 图片标题
    \label{fig:single}  % 标签(用于交叉引用:\ref{fig:single})
\end{figure}
\end{comment}


\begin{comment}
%两个图并排
\begin{figure}[H]
    \centering
    % 子图1:宽度占页面的45%(左右留空)
    \begin{subfigure}[b]{0.45\textwidth}  % [b]表示底部对齐
        \centering
        \includegraphics[width=\textwidth]{数学建模image/屏幕截图2025-08-04185706.png}  % 宽度=子图宽度
        \caption{子图1(如正面视图)}  % 子标题
        \label{fig:sub1}  % 子图标签
    \end{subfigure}
    \hspace{0.05\textwidth}  % 两图间距(5%页面宽度)
    % 子图2
    \begin{subfigure}[b]{0.45\textwidth}
        \centering
        \includegraphics[width=\textwidth]{数学建模image/屏幕截图2025-08-04185706.png}
        \caption{子图2(如侧面视图)}
        \label{fig:sub2}
    \end{subfigure}
    \caption{两张图片并排展示(整体标题)}  % 整体标题
    \label{fig:two}  % 整体标签
\end{figure}
\end{comment}


\begin{comment}
\begin{table}[htbp]
    \centering
    \begin{tabular}{ccccccccc}
        \toprule  % 顶部粗线
        作物名称                   & 地块类型    & 种植季次    & 亩产量     & 亩成本     & 销售单价    & 单位成本    & 边际收入    & 性价比     \\
        \midrule  % 表头与内容间的细线
        \cellcolor{blue!25}榆黄菇 & 普通大棚    & 第二季     & 5000    & 3000    & 57.5    & 0.60    & 95.8300 & 56.90   \\
        香菇                     & 普通大棚    & 第二季     & 4000    & 2000    & 19      & 0.50    & 38.0000 & 18.50   \\
        黄瓜                     & 普通大棚    & 第一季     & 15000   & 3500    & 7       & 0.23    & 30.0000 & 6.77    \\
        黄瓜                     & 智慧大棚    & 第二季     & 13500   & 3850    & 8.4     & 0.29    & 29.4500 & 8.11    \\
        芹菜                     & 水浇地     & 第一季     & 5500    & 900     & 4       & 0.16    & 24.4400 & 3.84    \\
        $\dots$                & $\dots$ & $\dots$ & $\dots$ & $\dots$ & $\dots$ & $\dots$ & $\dots$ & $\dots$ \\
        红薯                     & 梯田      & 单季      & 2100    & 2000    & 3.25    & 0.95    & 3.4125  & 2.30    \\
        黄豆                     & 平旱地     & 单季      & 400     & 400     & 3.25    & 1.00    & 3.2500  & 2.25    \\
        红薯                     & 山坡地     & 单季      & 2000    & 2000    & 3.25    & 1.00    & 3.2500  & 2.25    \\
        黄豆                     & 梯田      & 单季      & 380     & 400     & 3.25    & 1.05    & 3.0875  & 2.20    \\
        黄豆                     & 山坡地     & 单季      & 360     & 400     & 3.25    & 1.11    & 2.9250  & 2.14    \\
        \bottomrule  % 底部粗线
    \end{tabular}
    \caption{农作物相关数据(美观版)}
    \label{tab:crops_booktabs}
\end{table}

\end{comment}



\begin{comment}

\begin{table}[htbp]
    \centering
    \begin{tabular*}{\linewidth}{@{\extracolsep{\fill}}c c c}
        \toprule  % 顶部粗线
        作物名称    & 地块类型    & 种植季次    \\
        \midrule  % 表头与内容间的细线
        榆黄菇     & 普通大棚    & 第二季     \\
        香菇      & 普通大棚    & 第二季     \\
        黄瓜      & 普通大棚    & 第一季     \\
        黄瓜      & 智慧大棚    & 第二季     \\
        芹菜      & 水浇地     & 第一季     \\
        $\dots$ & $\dots$ & $\dots$ \\
        红薯      & 梯田      & 单季      \\
        黄豆      & 平旱地     & 单季      \\
        红薯      & 山坡地     & 单季      \\
        黄豆      & 梯田      & 单季      \\
        黄豆      & 山坡地     & 单季      \\
        \bottomrule  % 底部粗线
    \end{tabular*}
    \caption{农作物相关数据(美观版)}
    \label{tab:crops_booktabs}
\end{table}

\end{comment}
