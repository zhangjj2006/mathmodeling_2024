\documentclass{article}
\usepackage{forest,ctex,amsmath,color,xcolor,colortbl,graphicx,geometry,comment,listings,hyperref,booktabs,tabularx,wrapfig,caption,calc,mhchem,fix-cm,tocloft,titlesec,setspace,fancyhdr,booktabs,float,subcaption}
\usepackage{lastpage}
\usepackage[absolute,overlay]{textpos}  % 绝对定位
\setlength{\parindent}{2em} 
% ========== 标题格式定制(titlesec宏包) ========== %
\renewcommand{\thesection}{\chinese{section}}  % 一级标题编号:第X章
\renewcommand{\thesubsection}{\arabic{section}.\arabic{subsection}} % 二级:1.1
\renewcommand{\thesubsubsection}{\arabic{section}.\arabic{subsection}.\arabic{subsubsection}} % 三级:1.1.1
% 一级标题:三号黑体,居中,上下各空12pt(≈1行)
\titleformat{\section}
  [block]                  % 标题格式:display(带上下间距)
  {\centering \zihao{3} \heiti}  % 居中,三号黑体
  {\Large \thesection}       % 编号格式(Large字号,如“第一章”)
  {12pt}                     % 编号与标题的间距
  {}                         % 标题内容(无额外格式)
\titlespacing{\section}{0pt}{30pt}{30pt}  % 上间距12pt,下间距12pt

% 二级标题:四号黑体,缩进,编号显示(如“1.1”)
\titleformat{\subsection}
  {\zihao{4} \heiti}         % 四号黑体
  {\thesubsection}           % 编号(如“1.1”)
  {1em}                      % 编号与标题的间距
  {}
\titlespacing{\subsection}{0em}{1.5ex plus .5ex}{1ex plus .2ex}  % 左缩进2em

% 三级标题:小四号黑体,缩进,编号显示(如“1.1.1”)
\titleformat{\subsubsection}
  {\zihao{-4}}        % 小四号黑体(\zihao{-4}对应小四号)
  {\thesubsubsection}        % 编号(如“1.1.1”)
  {1em}                      % 编号与标题的间距
  {}
\titlespacing{\subsubsection}{0em}{1.5ex plus .5ex}{1ex plus .2ex}  % 左缩进2em


\geometry{left=2.5cm,right=2.5cm,top=2.5cm,bottom=2.5cm}
%\setlength{\parindent}{2pt} 
\date{}
\begin{document}
\vspace{-30pt}
\begin{center}
    \Large \textbf{数学建模比赛}  % 原字体可能是\Large,改为\large缩小
\end{center}
% 标题与“摘要”之间减少空白
\vspace{-13pt}
\begin{center}
    \large \textbf{摘要}  % 原字体可能是\large,改为\normalsize缩小
\end{center}
\qquad 本文研究111    问题

针对问题一

针对问题二

针对问题三

\textbf{关键词:}
\newpage
\section{\textbf{问题重述}}
\subsection{\textbf{问题背景}}
......
\subsection{\textbf{问题提出}}
。。。

\textbf{问题一:}

\textbf{问题二:}

\textbf{问题三:}

\end{document}

\begin{comment}

% 分段函数 Y_{tijk}
\[
    Y_{tijk} =
    \begin{cases}
        0, & \text{第 } i \text{ 季度未在第 } j \text{ 块地上种植物 } k \\
        1, & \text{第 } i \text{ 季度在第 } j \text{ 块地上种植物 } k
    \end{cases}
    \tag{4}
\]
\end{comment}


\begin{comment}
%公式
\begin{gather}
    X_{tijk} \leq M \cdot Y_{tijk} \quad \forall t,i,j,k \tag{5} \\
    X_{tijk} \geq 0.01 \cdot Y_{tijk} \quad \forall t,i,j,k \tag{6}
\end{gather}
\end{comment}


\begin{comment}

% 单个图片
\begin{figure}[H]  % [H]表示强制当前位置(可选参数:h=此处,t=顶部,b=底部,p=单独页)
    \centering  % 图片居中
    % 插入图片:width=0.8\textwidth 表示占页面宽度的80%(可调整)
    \includegraphics[width=0.8\textwidth]{数学建模image/屏幕截图2025-08-04185706.png}  % 替换为实际图片路径
    \caption{单张示例图片(如实验装置图)}  % 图片标题
    \label{fig:single}  % 标签(用于交叉引用:\ref{fig:single})
\end{figure}
\end{comment}


\begin{comment}
%两个图并排
\begin{figure}[H]
    \centering
    % 子图1:宽度占页面的45%(左右留空)
    \begin{subfigure}[b]{0.45\textwidth}  % [b]表示底部对齐
        \centering
        \includegraphics[width=\textwidth]{数学建模image/屏幕截图2025-08-04185706.png}  % 宽度=子图宽度
        \caption{子图1(如正面视图)}  % 子标题
        \label{fig:sub1}  % 子图标签
    \end{subfigure}
    \hspace{0.05\textwidth}  % 两图间距(5%页面宽度)
    % 子图2
    \begin{subfigure}[b]{0.45\textwidth}
        \centering
        \includegraphics[width=\textwidth]{数学建模image/屏幕截图2025-08-04185706.png}
        \caption{子图2(如侧面视图)}
        \label{fig:sub2}
    \end{subfigure}
    \caption{两张图片并排展示(整体标题)}  % 整体标题
    \label{fig:two}  % 整体标签
\end{figure}
\end{comment}


\begin{comment}
\begin{table}[htbp]
    \centering
    \begin{tabular}{ccccccccc}
        \toprule  % 顶部粗线
        作物名称                   & 地块类型    & 种植季次    & 亩产量     & 亩成本     & 销售单价    & 单位成本    & 边际收入    & 性价比     \\
        \midrule  % 表头与内容间的细线
        \cellcolor{blue!25}榆黄菇 & 普通大棚    & 第二季     & 5000    & 3000    & 57.5    & 0.60    & 95.8300 & 56.90   \\
        香菇                     & 普通大棚    & 第二季     & 4000    & 2000    & 19      & 0.50    & 38.0000 & 18.50   \\
        黄瓜                     & 普通大棚    & 第一季     & 15000   & 3500    & 7       & 0.23    & 30.0000 & 6.77    \\
        黄瓜                     & 智慧大棚    & 第二季     & 13500   & 3850    & 8.4     & 0.29    & 29.4500 & 8.11    \\
        芹菜                     & 水浇地     & 第一季     & 5500    & 900     & 4       & 0.16    & 24.4400 & 3.84    \\
        $\dots$                & $\dots$ & $\dots$ & $\dots$ & $\dots$ & $\dots$ & $\dots$ & $\dots$ & $\dots$ \\
        红薯                     & 梯田      & 单季      & 2100    & 2000    & 3.25    & 0.95    & 3.4125  & 2.30    \\
        黄豆                     & 平旱地     & 单季      & 400     & 400     & 3.25    & 1.00    & 3.2500  & 2.25    \\
        红薯                     & 山坡地     & 单季      & 2000    & 2000    & 3.25    & 1.00    & 3.2500  & 2.25    \\
        黄豆                     & 梯田      & 单季      & 380     & 400     & 3.25    & 1.05    & 3.0875  & 2.20    \\
        黄豆                     & 山坡地     & 单季      & 360     & 400     & 3.25    & 1.11    & 2.9250  & 2.14    \\
        \bottomrule  % 底部粗线
    \end{tabular}
    \caption{农作物相关数据(美观版)}
    \label{tab:crops_booktabs}
\end{table}

\end{comment}



\begin{comment}

\begin{table}[htbp]
    \centering
    \begin{tabular*}{\linewidth}{@{\extracolsep{\fill}}c c c}
        \toprule  % 顶部粗线
        作物名称    & 地块类型    & 种植季次    \\
        \midrule  % 表头与内容间的细线
        榆黄菇     & 普通大棚    & 第二季     \\
        香菇      & 普通大棚    & 第二季     \\
        黄瓜      & 普通大棚    & 第一季     \\
        黄瓜      & 智慧大棚    & 第二季     \\
        芹菜      & 水浇地     & 第一季     \\
        $\dots$ & $\dots$ & $\dots$ \\
        红薯      & 梯田      & 单季      \\
        黄豆      & 平旱地     & 单季      \\
        红薯      & 山坡地     & 单季      \\
        黄豆      & 梯田      & 单季      \\
        黄豆      & 山坡地     & 单季      \\
        \bottomrule  % 底部粗线
    \end{tabular*}
    \caption{农作物相关数据(美观版)}
    \label{tab:crops_booktabs}
\end{table}

\end{comment}
