\documentclass{article}
\usepackage{forest,ctex,amsmath,color,xcolor,colortbl,graphicx,geometry,comment,listings,hyperref,booktabs,tabularx,wrapfig,caption,calc,mhchem,fix-cm,tocloft,titlesec,setspace,fancyhdr,booktabs,float,subcaption}
\usepackage{lastpage}
\usepackage[absolute,overlay]{textpos}  % 绝对定位
\setlength{\parindent}{2em} 
% ========== 标题格式定制(titlesec宏包) ========== %
\renewcommand{\thesection}{\chinese{section}}  % 一级标题编号:第X章
\renewcommand{\thesubsection}{\arabic{section}.\arabic{subsection}} % 二级:1.1
\renewcommand{\thesubsubsection}{\arabic{section}.\arabic{subsection}.\arabic{subsubsection}} % 三级:1.1.1
% 一级标题:三号黑体,居中,上下各空12pt(≈1行)
\titleformat{\section}
  [block]                  % 标题格式:display(带上下间距)
  {\centering \zihao{3} \heiti}  % 居中,三号黑体
  {\Large \thesection}       % 编号格式(Large字号,如“第一章”)
  {12pt}                     % 编号与标题的间距
  {}                         % 标题内容(无额外格式)
\titlespacing{\section}{0pt}{30pt}{30pt}  % 上间距12pt,下间距12pt

% 二级标题:四号黑体,缩进,编号显示(如“1.1”)
\titleformat{\subsection}
  {\zihao{4} \heiti}         % 四号黑体
  {\thesubsection}           % 编号(如“1.1”)
  {1em}                      % 编号与标题的间距
  {}
\titlespacing{\subsection}{0em}{1.5ex plus .5ex}{1ex plus .2ex}  % 左缩进2em

% 三级标题:小四号黑体,缩进,编号显示(如“1.1.1”)
\titleformat{\subsubsection}
  {\zihao{-4}}        % 小四号黑体(\zihao{-4}对应小四号)
  {\thesubsubsection}        % 编号(如“1.1.1”)
  {1em}                      % 编号与标题的间距
  {}
\titlespacing{\subsubsection}{0em}{1.5ex plus .5ex}{1ex plus .2ex}  % 左缩进2em


\geometry{left=2.5cm,right=2.5cm,top=2.5cm,bottom=2.5cm}
%\setlength{\parindent}{2pt} 
\date{}
\begin{document}
\vspace{-30pt}
\begin{center}
    \Large \textbf{数学建模比赛}  % 原字体可能是\Large,改为\large缩小
\end{center}
% 标题与“摘要”之间减少空白
\vspace{-13pt}
\begin{center}
    \large \textbf{摘要}  % 原字体可能是\large,改为\normalsize缩小
\end{center}
\qquad 本文针对NIPT(无创产前检测)技术中最佳检测时点选择与胎儿异常判定的问题,通过建立数学模型,对
孕妇进行合理分组并确定最佳检测策略,以最小化潜在风险并提高检测准确性。我们综合利用了数理统计等方法,
系统地解决了题目提出的四个问题。

针对问题一,首先处理表格中的数据,把检测次数不足四次的孕妇代码剔除并把检测孕周换算成天数,方便后使之后
的线性回归计算更加可靠。之后剩下的每个人的y染色体浓度与检测孕周(天数)进行\textbf{线性回归}计算方程与
$R^2$,考察到整体y染色体浓度与检测孕周的\textbf{线性拟合程度都很高}。之后对于这些人,求出检测的时间中BMI增
长速率。将每个人的线性回归程斜率与进行相关性分析,得到了一套比较显著可靠的关系模型。最后引入\textbf{斯皮尔曼系数}进行检验

针对问题二

针对问题三

针对问题四

\textbf{关键词:线性回归,斯皮尔曼系数}
\newpage
\section{\textbf{问题重述}}
\subsection{\textbf{问题背景}}
NIPT(Non-invasive Prenatal Testing,无创产前检测)作为一项革命性的产前筛查技术,通过采集孕
妇外周血中的胎儿游离DNA进行测序分析,能够有效评估胎儿常见染色体非整倍体异常的风险[1]。该技术具
有无创、安全、准确性高等特点,已成为临床产前筛查的重要手段。

在实际临床应用中,NIPT检测的准确性受到多种因素影响,其中胎儿游离DNA浓度(特别是Y染色体浓度对
于男胎)是关键因素之一。临床实践表明,胎儿Y染色体浓度与孕妇孕周数及身体质量指数(BMI)存在显著
相关性[2]。同时,检测时机的选择对于尽早发现胎儿异常、降低临床风险至关重要:早期发现(12周以内
)风险较低,中期发现(13-27周)风险较高,而晚期发现(28周以后)风险极高。

目前临床通常根据孕妇BMI值进行简单分组并确定统一的检测时点,但这种方法未能充分考虑孕妇年龄、体
重等个体差异,可能导致部分孕妇错过最佳检测时机,增加临床风险。因此,需要建立更加科学、个性化的
NIPT时点选择模型,为不同特征的孕妇群体制定最优检测策略。

\subsection{\textbf{问题要求}}
附件提供了某地区(大多为高BMI)孕妇的NIPT检测数据,包括孕妇年龄、BMI、孕周数、胎儿染色体浓度、
Z值、GC含量、读段数等相关指标。现需要根据这些数据建立数学模型,解决以下问题:

\textbf{问题一:}基于附件中的数据,分析胎儿Y染色体浓度与孕妇孕周数、BMI等指标的相关特性,建
立合适的数学模型描述它们之间的关系,并对模型的显著性进行统计检验。

\textbf{问题二:}临床证明男胎孕妇的BMI是影响胎儿Y染色体浓度达标时间(浓度≥4\%的最早时间)的
主要因素。请对男胎孕妇的BMI进行合理分组,确定每组的BMI区间和最佳NIPT检测时点,使得孕妇的潜在
风险最小,并分析检测误差对结果的影响。

\textbf{问题三:}综合考虑体重、年龄等多种因素对男胎Y染色体浓度达标时间的影响,同时考虑检测误
差和胎儿Y染色体浓度达标比例,根据男胎孕妇的BMI进行合理分组,确定每组的最佳NIPT检测时点,使孕
妇潜在风险最小,并分析检测误差对结果的影响。

\textbf{问题四:}针对女胎异常的判定问题,以女胎孕妇的21号、18号和13号染色体非整倍体为判定结
果,综合考虑X染色体及上述染色体的Z值、GC含量、读段数及相关比例、BMI等因素,建立女胎异常的判定
模型和方法。

\section{\textbf{问题分析}}
\subsection{\textbf{问题一的分析}}
针对问题一,我们需要先对附件中的数据进行处理,为之后对每个人进行线性回归分析,剔除掉同一个人检
测次数小于等于三的孕妇代码,使拟合结果更加可靠。之后对于每一个人,以y染色体为y轴,检测孕周(天
数)为x轴进行线性回归计算,并求$R^2$确定拟合程度。之后对与每个孕妇代码求出BMI增长率,对比斜率
与BMI增长率,检测其相关性。

\textbf{问题一:}

\textbf{问题二:}

\textbf{问题三:}

\section{\textbf{模型假设}}
1.假设附件所提供的孕妇NIPT检测数据真实、准确,且数据样本足以反映胎儿染色体浓度与孕妇孕周、BMI等指标间的统计规律。

2.假设题目中给出的Y染色体浓度4\% 的临界值是可靠且普适的。

3.假设具有相似特征(如处于同一BMI区间)的孕妇群体,其胎儿Y染色体浓度的增长规律和达标时间具有相似的统计特征。

4.假设题目所提供的特征(包括但不限于X、21、18、13号染色体的Z值、GC含量、读段数比例及孕妇BMI等)包含了足以有效判别女胎染色体是否异常的信息。

\section{\textbf{符号说明}}
\begin{table}[htbp]
    \centering
    % 列格式:第一列居中+固定宽3cm,第二列居中
    \begin{tabular*}{\linewidth}{@{\extracolsep{\fill}}>{\centering\arraybackslash}p{3cm} c}
        \toprule  % 顶部粗线
        符号 & 说明 \\
        \midrule  % 表头与内容间的细线
        $k_i$ & 孕妇代码为$i$的个体Y染色体浓度与孕周线性回归方程斜率 \\
        $c_{ij}$ & 孕妇代码为$i$的个体第$j$次检测得到的Y染色体浓度\\
        $t_ij$ & 孕妇代码为$i$的个体第$j$次检测的检测孕周(转化为天数)\\
        $b_i$ & 孕妇代码为$i$的个体BMI增长率\\
        $\overline{b}$ & 所有孕妇个体BMI增长率的平均数\\
        \bottomrule  % 底部粗线
    \end{tabular*}
    \label{tab:symbols}
\end{table}

\section{问题一模型建立求解}
\subsection{模型建立思路}
问题一要求分析胎儿Y染色体浓度与孕妇孕周数和BMI等指标的相关特性。为精确刻画个体增长规律并避免群
体平均带来的偏差,我们决定采用\textbf{基于个体时间序列的线性拟合方法}。核心思路是:首先对数据
进行筛选,保证每个个体的数据点足以支持回归分析;然后为每一位符合条件的孕妇单独建立Y染色体浓度
随孕周(天数)变化的线性回归模型,以拟合斜率量化其增长速率,从而探究胎儿 Y 染色体浓度与孕妇的孕周数的关系;然后为每一位孕妇建立BMI随孕周(天数)变化的线性回归模型,并将BMI增长率与 Y 染色体浓度的增长率进行相关性分析,从而探究胎儿 Y 染色体浓度与孕妇的BMI的关系;最后综合两者,得出胎儿Y染色体浓度与孕妇孕周数和BMI的相关特性。

\subsection{数据预处理}
附件中的数据存在个别孕妇检测次数过少的情况,这会导致线性回归结果不可靠。为确保模型稳定性,我们
设定了数据筛选条件:仅保留检测次数大于3次的孕妇数据。

对于计算出来的结果,我们剔除掉以下拟合直线:

1.$R^2$小于0.5。此时拟合效果差。

2.斜率小于0此时Y染色体。此时浓度呈现随时间下降趋势,而经过收集资料,我们发现孕妇体内胎儿Y染色
体浓度的正常趋势是:孕早期(10-12 周)从无法检出到逐步升高→孕中期(12-22 周)稳步升至峰值→孕
晚期(>22 周)进入稳定平台期(允许轻微波动,但始终可检出) 。

经过预处理,原始数据中共包含261名孕妇的记录,其中符合要求的有效孕妇样本为178名。
\subsection{个体线性回归模型}
对于每一位有效孕妇个体 $i$,以其多次检测的孕周(转换为天数$t_{ij}$)为自变量,对应的Y
染色体浓度($c_{ij}$)为因变量,建立一元线性回归模型:
%公式
\begin{gather}
    c_{ij}=k_i*t_{ij}+b_i+\epsilon_{ij} \tag{1}
\end{gather}

其中:$k_i$ 表示第 $i$ 位孕妇的胎儿Y染色体浓度的日增长率,这是我们关注的核心参数;
$b_i$ 为截距项;$\epsilon_{ij}$ 为随机误差项。

我们采用最小二乘法进行参数估计,并计算确定系数 $R^2_i$ 以评估拟合优度

\begin{comment}
% 单个图片
\begin{figure}[H]  % [H]表示强制当前位置(可选参数:h=此处,t=顶部,b=底部,p=单独页)
    \centering  % 图片居中
    % 插入图片:width=0.8\textwidth 表示占页面宽度的80%(可调整)
    \includegraphics[width=0.5\textwidth]{graph/R2sandian.png}  % 替换为实际图片路径
    \caption{样例线性回归拟合程度 R\^2}  % 图片标题
    \label{fig:single}  % 标签(用于交叉引用:\ref{fig:single})
\end{figure}
\end{comment}
\subsection{BMI与增长速率的全局相关性分析}
为探究BMI是否导致了上述增长速率的差异,我们计算了每位孕妇的BMI增长率个体Y染色体浓度增长率 $k_i$ 的相关性。

我们绘制了标准化BMI增长速率与斜率 $k_i$ 的散点图和对两者都进行归一化之后的散点图,如下
%两个图并排
\begin{figure}[H]
    \centering
    % 子图1:宽度占页面的45%(左右留空)
    \begin{subfigure}[b]{0.45\textwidth}  % [b]表示底部对齐
        \centering
        \includegraphics[width=\textwidth]{graph/biaozhunhua.png}  % 宽度=子图宽度
        \caption{标准化BMI增长速率与斜率的散点图}  % 子标题
        \label{fig:sub1}  % 子图标签
    \end{subfigure}
    \hspace{0.05\textwidth}  % 两图间距(5%页面宽度)
    % 子图2
    \begin{subfigure}[b]{0.45\textwidth}
        \centering
        \includegraphics[width=\textwidth]{graph/guiyihua.png}
        \caption{归一化BMI增长速率与归一化斜率的散点图}
        \label{fig:sub2}
    \end{subfigure}
    \label{fig:two}  % 整体标签
\end{figure}
我们可以看到这两组点的吻合程度极高,由此得出结论:BMI增长率个体Y染色体浓度增长率有较高关联性。

之后,我们采用\textbf{Pearson相关系数} $r$ 来衡量二者之间的线性相关强度,并计算其显著性p值
以判断该相关性是否由随机因素引起。计算公式如下:
\textbf{Pearson相关系数计算公式:}
%公式
\begin{gather}
    r=\frac{\sum_{i=1}^{n}(b_i-\overline{b})(k_i-\overline{k})}{\sqrt{\sum_{i=1}^{n}(b_i-\overline{b})^2}\sqrt{\sum_{i=1}^{n}(k_i-\overline{k})^2}} \tag{2}
\end{gather}
其中,$n$ 为有效孕妇样本数,$b_i$ 和 $k_i$ 分别为第 $i$ 位孕妇的初始BMI和Y染色体浓度日增长率,$\overline{b}$ 和 $\bar{k}$ 分别为它们的样本均值。

\textbf{显著性检验(t检验)统计量及p值计算公式:}

为检验相关系数的显著性,构建t统计量:
\begin{gather}
    t=r\sqrt{\frac{n-2}{1-r^2}}\tag{3}
\end{gather}

我们通过上述公式计算得到r值为0.2795,p值为0.000165,满足$p < 0.001$。因此,基于统计结果,我们有极大把握认为两者具有显著相关性。
这表明,\textbf{BMI更高的孕妇,其胎儿Y染色体浓度随孕周增长的速度更快。}这一结论与临床经验相符,为临床上进行个性化NIPT时点预测提供了重要的数学依据。
\end{document}

\begin{comment}

% 分段函数 Y_{tijk}
\[
    Y_{tijk} =
    \begin{cases}
        0, & \text{第 } i \text{ 季度未在第 } j \text{ 块地上种植物 } k \\
        1, & \text{第 } i \text{ 季度在第 } j \text{ 块地上种植物 } k
    \end{cases}
    \tag{4}
\]
\end{comment}


\begin{comment}
%公式
\begin{gather}
    X_{tijk} \leq M \cdot Y_{tijk} \quad \forall t,i,j,k \tag{5} \\
    X_{tijk} \geq 0.01 \cdot Y_{tijk} \quad \forall t,i,j,k \tag{6}
\end{gather}
\end{comment}


\begin{comment}

% 单个图片
\begin{figure}[H]  % [H]表示强制当前位置(可选参数:h=此处,t=顶部,b=底部,p=单独页)
    \centering  % 图片居中
    % 插入图片:width=0.8\textwidth 表示占页面宽度的80%(可调整)
    \includegraphics[width=0.8\textwidth]{数学建模image/屏幕截图2025-08-04185706.png}  % 替换为实际图片路径
    \caption{单张示例图片(如实验装置图)}  % 图片标题
    \label{fig:single}  % 标签(用于交叉引用:\ref{fig:single})
\end{figure}
\end{comment}


\begin{comment}
%两个图并排
\begin{figure}[H]
    \centering
    % 子图1:宽度占页面的45%(左右留空)
    \begin{subfigure}[b]{0.45\textwidth}  % [b]表示底部对齐
        \centering
        \includegraphics[width=\textwidth]{数学建模image/屏幕截图2025-08-04185706.png}  % 宽度=子图宽度
        \caption{子图1(如正面视图)}  % 子标题
        \label{fig:sub1}  % 子图标签
    \end{subfigure}
    \hspace{0.05\textwidth}  % 两图间距(5%页面宽度)
    % 子图2
    \begin{subfigure}[b]{0.45\textwidth}
        \centering
        \includegraphics[width=\textwidth]{数学建模image/屏幕截图2025-08-04185706.png}
        \caption{子图2(如侧面视图)}
        \label{fig:sub2}
    \end{subfigure}
    \caption{两张图片并排展示(整体标题)}  % 整体标题
    \label{fig:two}  % 整体标签
\end{figure}
\end{comment}


\begin{comment}
\begin{table}[htbp]
    \centering
    \begin{tabular}{ccccccccc}
        \toprule  % 顶部粗线
        作物名称                   & 地块类型    & 种植季次    & 亩产量     & 亩成本     & 销售单价    & 单位成本    & 边际收入    & 性价比     \\
        \midrule  % 表头与内容间的细线
        \cellcolor{blue!25}榆黄菇 & 普通大棚    & 第二季     & 5000    & 3000    & 57.5    & 0.60    & 95.8300 & 56.90   \\
        香菇                     & 普通大棚    & 第二季     & 4000    & 2000    & 19      & 0.50    & 38.0000 & 18.50   \\
        黄瓜                     & 普通大棚    & 第一季     & 15000   & 3500    & 7       & 0.23    & 30.0000 & 6.77    \\
        黄瓜                     & 智慧大棚    & 第二季     & 13500   & 3850    & 8.4     & 0.29    & 29.4500 & 8.11    \\
        芹菜                     & 水浇地     & 第一季     & 5500    & 900     & 4       & 0.16    & 24.4400 & 3.84    \\
        $\dots$                & $\dots$ & $\dots$ & $\dots$ & $\dots$ & $\dots$ & $\dots$ & $\dots$ & $\dots$ \\
        红薯                     & 梯田      & 单季      & 2100    & 2000    & 3.25    & 0.95    & 3.4125  & 2.30    \\
        黄豆                     & 平旱地     & 单季      & 400     & 400     & 3.25    & 1.00    & 3.2500  & 2.25    \\
        红薯                     & 山坡地     & 单季      & 2000    & 2000    & 3.25    & 1.00    & 3.2500  & 2.25    \\
        黄豆                     & 梯田      & 单季      & 380     & 400     & 3.25    & 1.05    & 3.0875  & 2.20    \\
        黄豆                     & 山坡地     & 单季      & 360     & 400     & 3.25    & 1.11    & 2.9250  & 2.14    \\
        \bottomrule  % 底部粗线
    \end{tabular}
    \caption{农作物相关数据(美观版)}
    \label{tab:crops_booktabs}
\end{table}

\end{comment}



\begin{comment}

\begin{table}[htbp]
    \centering
    \begin{tabular*}{\linewidth}{@{\extracolsep{\fill}}c c c}
        \toprule  % 顶部粗线
        作物名称    & 地块类型    & 种植季次    \\
        \midrule  % 表头与内容间的细线
        榆黄菇     & 普通大棚    & 第二季     \\
        香菇      & 普通大棚    & 第二季     \\
        黄瓜      & 普通大棚    & 第一季     \\
        黄瓜      & 智慧大棚    & 第二季     \\
        芹菜      & 水浇地     & 第一季     \\
        $\dots$ & $\dots$ & $\dots$ \\
        红薯      & 梯田      & 单季      \\
        黄豆      & 平旱地     & 单季      \\
        红薯      & 山坡地     & 单季      \\
        黄豆      & 梯田      & 单季      \\
        黄豆      & 山坡地     & 单季      \\
        \bottomrule  % 底部粗线
    \end{tabular*}
    \caption{农作物相关数据(美观版)}
    \label{tab:crops_booktabs}
\end{table}

\end{comment}
